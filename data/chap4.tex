\chapter{实验}
\label{cha:eval}
    我们的测试实验主要分为三部分,第一部分对调度预测进行测试,第二部分对性能模型进行测试,第三部分对整体模型进行测试。
    
\section{测试集}
    对第一部分测试和第三部分测试,我们采用自行创建的数据流图和现实存在的神经网络两类模型进行测试。
    
    自建数据包含三个测试:
    
    1. 简单矩阵级联乘法计算测试:测试多个矩阵乘法级联的情况。ref所示。
    
    2. 并行调度测试:测试可以并行的计算节点的调度情况。如图ref所示。

    3. 多操作资源抢占测试:测试多个不同操作抢占硬件资源的情况。如图ref所示。

    神经网络测试包含两个测试:
    
    1. LeNet-5:作为规模较小的卷积神经网络。
    
    2. AlexNet:作为规模较大的卷积神经网络。
    
    由于卷积神经网络的结构没有较为大的变化,因此选取这两个测试可以较好的说明我们的模型的预测能力。

\section{调度预测}
    这一节我将针对调度预测模块进行测试。我们的目标是测试调度模拟能够多大程度恢复任务真实的运行状况。因此,我们的测试方法是,首先正常运行测试,拿到测试的时间轴数据,得到每个节点的运行时间。之后将这一数据作为性能模型代入调度模拟器。这时用得到的预测结果和实际运行时间进行比较,就可以说明调度模拟器的性能。
    
\section{性能模型}
    这一节我将针对性能模型模块进行测试。我的测试方法是,针对每个操作随机生成若干组测试点,使用预测结果和实际运行时间进行比较。
    
\section{性能模型}
    这一节我将对整体性能模型进行测试。测试方法就是直接测试第一节中的测试,得到运行时间,和模型的预测时间做对比,这样能够最为直观地比较性能模型的实际预测效果。

