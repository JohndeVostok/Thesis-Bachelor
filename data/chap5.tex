\chapter{结论}
\section{总结}
\label{cha:conc}
    为了解决预测深度神经网络运行时间的问题,我们提出了基于“预测-调度”框架的深度神经网络性能预测模型。我们通过静态分析神经网络的结构,对数据流图调度运行过程进行模拟,再结合预先训练的操作性能模型的方式实现了深度神经网络的静态细粒度预测。
    
    在调度模拟层面,我们非常好地还原了TensorFLow中操作的调度模式。在单机CPU情况下,我们的模型能根据追踪数据还原深度神经网络运行时间。在单机多GPU的环境下,我们的模型能够很好地还原TensorFlow中对多GPU的调度过程。
    
    在性能模型方面,我们对矩阵乘法、二维卷积、局部响应归一化等函数进行了性能建模。其中矩阵乘法和局部响应归一化函数得到了比较好的建模,而二维卷积在GPU上的性能模型性能不够好,这一部分拖累了综合测试中整体的预测准确度。
    
\section{未来工作}
    我们的模型还有比较大的优化空间。
    
    在性能建模方面,在我们建立的几个性能模型中,包含部分近似项,后续工作可以通过更多的性能测试拿到更多的标签数据。在更多数据的基础上,可以考虑使用深度神经网络帮助获得更准确的性能预测模型。
    
    在操作选择方面,我们仅对矩阵乘法和二维卷积等操作进行了性能建模。在网络规模较小运行次数较多的时候,其他操作的时间可能会占到运行时间的主要部分,这时候我们的模型会出现明显的精度损失。
    
    调度模拟方面,我们的测试仅仅局限于单机。但是由TensorFlow自身结构决定,多机的情况我们的模型实际也可以进行解决,仅需要补充多机之间数据传输的性能模型即可。
    
    调度优化方面,在我们的结构下,我们可以尝试对TensorFlow进行调度优化。目前TensorFlow的开源版本中基本上没有进行操作粒度上的调度优化,而我们的模拟工具可以在不实际运行神经网络的情况下提供较为准确的性能数据,方便调度优化算法的设计模拟。
    
    测试环境方面,后续可以添加更多不同CPU、GPU下操作的性能模型,方便我们的模型在云平台上使用。
    
\section{展望}

    我们可以认为,“预测-调度”结构给我们带来了比较好的预测精度,这一结构可以应用在更多的数据流图计算结构,如Spark\cite{rdd}等系统。我们衷心希望我们的想法能够启发更多人使用我们模型背后的想法来解决更多的数据流图调度问题。
    
