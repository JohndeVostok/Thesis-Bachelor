\chapter{引言}
\label{cha:intro}

    人工智能已经成为了人们生活中不可分割的一部分。而人工智能的背后就是深度神经网络。随着云计算的发展,深度神经网络的部署也从本地逐渐移动到了云平台。现在亚马逊的AWS、微软的Azure均提供了云上的机器学习服务。在云计算中,可预测性是一个非常重要的指标。\cite{cloud, serverless}在云上执行任务时,由于用户需要根据使用时间或是使用的计算资源进行付费,用户就需要在使用云服务前,对自己已有的机器学习任务有一定的性能评价,以便选择更适合的云服务,特别是对企业而言。因此,我们希望能实现一个工具,能够通过{\bfseries 静态分析}深度学习任务,预测任务在特定平台上的运行时间。

    深度神经网络的规模和计算量日益增长,因此深度神经网络的训练过程更多的使用各类加速器,如GPU、TPU等。在Facebook于2017年的工作中,他们使用了256块GPU在一个小时内训练ResNet-50。\cite{fb_imagenet}我们可以确定的是,将来深度神经网络一定会运行在异构平台上。通常,使用异构平台要求计算更加规整,以通过并行的方式快速执行任务。因此,常用的加速框架,都将计算任务以数据流图的形式组织。
    
    数据流图是一个有向无环图,用点表示计算任务,边表示数据流动。\cite{dataflow}得益于加速器的发展,数据流图现在的使用越来越广泛了。那么,我们可以有一个很简单的想法,如果我们能拿到数据流图上每个点的计算时间,结合系统运行数据流图的方式,我们就可以得到整个计算任务的运行时间了。
    
    深度学习训练中常用的框架TensorFlow\cite{tensorflow}就是基于数据流图实现的。其中所有的计算任务都是一个操作,操作作为计算任务在数据流图中以点的形式保存。而TensorFlow中的数据以张量的形式存储,并以边的形式被表示在数据流图中。
    
    因此,基于TensorFlow,我们建立了自己的性能预测工具。首先对TensorFlow中的操作建立性能模型,以通过静态数据预测每个操作的运行时间。之后根据需要预测的深度学习任务建立数据流图,在数据流图上模拟真实的运行情况进行调度,结合性能模型数据,我们就能还原出真实的运行过程以得到模型的运行时间。
    
    我们的工具能比较好地预测操作的运行时间,准确地还原深度学习任务的调度过程。与TensorFlow自带的性能预测工具cost\_analyzer相比,我们的工具最多能将误差降低到原先的1\%以下。文章的主要贡献如下:
    
    \begin{itemize}
        \setlength{\itemindent}{1em}
        \item 提出了基于“预测-调度”模式的性能预测框架
        \item 基于提出的预测框架建立性能测试工具,并达到比较好的预测效果
    \end{itemize}
    
    \begin{translationbib}
        \item sb
    \end{translationbib}