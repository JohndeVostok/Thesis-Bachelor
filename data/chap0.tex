\chapter{介绍}
\label{cha:intro}

\section{动机}
\label{sec:motivation}
    随着机器学习的发展,深度神经网络有了越来越广泛的应用,如在机器视觉,自然语言处理等领域。于此同时,深度神经网络的深度和规模都越来越大,因此,深度神经网络的运行需要{\bfseries 更大规模的运算设备以及更长的运行时间}。那么,选择哪种设备或知道在固定设备上,模型的运行时间,就成了一个重要的问题。

    运行机器学习应用有两种不同的模式。在本地设备运行或在云平台运行。通常,企业会有自己的设备集群,并配备相应的管理方式。这种情况下,计算资源成为了企业内部争抢的资源,这时如何进行任务层面的调度,就成为了一个非常重要的问题。而更准确的离线性能预测模型,能够给企业提供更好的调度效果,提高工作效率。
    
    另一方面,当资金比较缺乏,不能够自己组建设备,或需要的计算资源规模太大(如Facebook 2018年的工作一样,使用256块GPU在1小时内训练ImageNet)。这些情况下,在本地运行都不现实,因此,使用云计算平台成为了非常合适的解决方案。
    
    云计算平台上通常提供了现成的机器学习应用框架,如AWS Machine Learning, Azure Machine Learning等,他们提供了从现有模型导入到云计算平台运行的功能。而另一种方式是直接租用实例(如AWS EC2等)搭建机器学习框架进行运行。无论使用哪种方式,现有的云计算平台都是按照时间收费的。因此,预测模型在不同设备上的运行时间,能够帮助用户更好的规划资金,选择配置。
    
    而另一方面,现有的机器学习框架,如tensorflow等,在集群模式下,通常使用数据并行方式进行加速。而在模型并行模式下,由于缺少操作性能数据模型,不能得到每个操作较为准确的运行时间,限制了模型并行的细粒度调度。因此,在进行细粒度性能预测的过程中,模型能够给机器学习框架提供细粒度调度建议,提高模型并行的运行速度。
    
    综上所述,深度神经网络性能预测模型无论是对工业界还是学术界都很重要。

\section{相关工作}
\label{sec:related}

\subsection{TensorFlow}
    TensorFlow是一个端到端的开源机器学习平台,目前在机器学习领域广泛使用。用户在已经配置好的tensorFlow平台上,可以定义模型并运行,不需要过多的考虑硬件设备。TensorFlow的模型主要分为两部分,数据以及运算,数据以张量(tensor)的形式保存,运算以数据流图(data flow graph)的形式进行组织,TensorFlow在运行过程中,根据硬件设备对数据流图进行调度,并通过Stream Executor进行执行。

\subsection{数据流图}
    数据流图是应用程序并行运行的一种组织方式。通常,数据流图是一个有向无环图(DAG)。数据流图的点表示一个运算,即一个操作,而边表示数据路径。这种组织方式能够明确的显示运算之间的依赖关系,方便程序进行调度。在TensorFlow中,数据流图的节点表示一个操作,而在调度中,数据流图中的节点会被预先分配到某个节点的某个设备上,再通过Stream Executor进行执行。

\subsection{Stream Executor}
    Stream Executor是Google对CUDA和OpenCL进行的封装。它能够管理任务的并行执行,同时保证同一套代码能够运行在不同的设备上。TensorFlow使用的是Stream Executor的简化版本,而在TensorFlow中,所有操作都使用Stream Executor进行调度,因此用户可以不需要直接对设备进行管理,实际使用中,Tensorflow的Stream Executor模块在GPU运行时主要工作就是将矩阵乘法、二维卷积等操作映射到CUBlas或CUDNN的函数。

\section{我们的方法以及贡献}
    我们的工作在TensorFlow的基础上进行,目标是对用户在TensorFlow上定义的模型,进行静态分析,估算模型的训练运行时间,从而使用户能够对执行时间有更高的预估,从而选择更合适的云计算服务或进行更好的调度。
    
    预测方法是,模拟真实TensorFlow的运行过程,先对TensorFlow生成的计算图进行静态分析,得到计算图节点的调度结果,包括计算图节点的运行依赖关系,计算图节点的运行所在设备等。同时,根据每个节点预测的运行时间,模拟TensorFlow的真实调度过程,综合两部分数据,就可以得到模型的真实运行时间。
    
    模拟器部分,通过对TensorFlow运行过程的研究,运行过程主要分三部分。
    
    操作性能模拟部分,根据数据性能部分拿到的数据,我认为,机器学习程序运行的主要时间主要在以下几类操作上,只要对这些操作进行建模,即可得到模型大体的性能数据。
    
    本文的主要贡献,提出了一个能够在TensorFlow上预测模型训练时间的模型,得到部分操作的性能模型,为细粒度调度优化提供支持。
