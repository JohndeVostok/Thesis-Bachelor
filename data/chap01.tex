\chapter{介绍}
\label{cha:intro}

\section{动机}
\label{sec:motivation}
    随着机器学习的发展,深度神经网络有了越来越广泛的应用,如在机器视觉,自然语言处理等领域。于此同时,深度神经网络的深度和规模都越来越大,因此,深度神经网络的运行需要{\bfseries 更大规模的运算设备以及更长的运行时间}。那么,选择哪种设备或知道在固定设备上,模型的运行时间,就成了一个重要的问题。

    运行机器学习应用有两种不同的模式。在本地设备运行或在云平台运行。通常,企业会有自己的设备集群,并配备相应的管理方式。这种情况下,计算资源成为了企业内部争抢的资源,这时如何进行任务层面的调度,就成为了一个非常重要的问题。而更准确的离线性能预测模型,能够给企业提供更好的调度效果,提高工作效率。
    
    另一方面,当资金比较缺乏,不能够自己组建设备,或需要的计算资源规模太大(如Facebook 2018年的工作一样,使用256块GPU在1小时内训练ImageNet)。这些情况下,在本地运行都不现实,因此,使用云计算平台成为了非常合适的解决方案。
    
    云计算平台上通常提供了现成的机器学习应用框架,如AWS Machine Learning, Azure Machine Learning等,他们提供了从现有模型导入到云计算平台运行的功能。而另一种方式是直接租用实例(如AWS EC2等)搭建机器学习框架进行运行。无论使用哪种方式,现有的云计算平台都是按照时间收费的。因此,预测模型在不同设备上的运行时间,能够帮助用户更好的规划资金,选择配置。
    
    而另一方面,现有的机器学习框架,如tensorflow等,在集群模式下,通常使用数据并行方式进行加速。而在模型并行模式下,由于缺少操作性能数据模型,不能得到每个操作较为准确的运行时间,限制了模型并行的细粒度调度。因此,在进行细粒度性能预测的过程中,模型能够给机器学习框架提供细粒度调度建议,提高模型并行的运行速度。
    
    综上所述,深度神经网络性能预测模型无论是对工业界还是学术界都很重要。

\section{相关工作}
\label{sec:related}

\subsection{Tensorflow}
    深度学习应用通常运行在深度学习框架上,